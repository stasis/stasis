
%\documentclass[letterpaper,english]{article}

\documentclass[10pt,letterpaper,twocolumn,english]{article}

% This fixes the PDF font, whether or not pdflatex is used to compile the document...
\usepackage{pslatex} 

\usepackage[T1]{fontenc}
\usepackage[latin1]{inputenc}
\usepackage{graphicx}
\usepackage{xspace}

\usepackage{geometry,color}
\geometry{verbose,letterpaper,tmargin=1in,bmargin=1in,lmargin=0.75in,rmargin=0.75in}

\makeatletter

\usepackage{babel}

\newcommand{\yad}{LLADD\xspace}
\newcommand{\oasys}{Juicer\xspace}

\newcommand{\eab}[1]{\textcolor{red}{\bf EAB: #1}}
\newcommand{\rcs}[1]{\textcolor{green}{\bf RCS: #1}}
\newcommand{\mjd}[1]{\textcolor{blue}{\bf MJD: #1}}
\newcommand{\jsk}[1]{\textcolor{brown}{\bf JSK: #1}}

%% for space
%% \newcommand{\eab}[1]{}
%% \newcommand{\rcs}[1]{}
%% \newcommand{\mjd}[1]{}

\begin{document}
%\title{\vspace*{-36pt}Application-specific program optimizations\vspace*{-36pt}}
\title{\vspace*{-36pt}Application-specific program optimizations}
\author{Jimmy Kittiyachavalit \and Russell Sears}
\maketitle


%\subsection*{Abstract}


{\em 
Modern software systems are partitioned into modules in order to allow
development efforts to scale gracefully.  Typically, each of these
modules is designed to hide underlying complexity from other software
components, and provides a relatively simple interface so that
programmers that are unfamiliar with the implementation of the module
can reuse it easily. Often, performance is traded for ease of use, as
the client code is unaware of potentials for optimization that the
library's internals present, and the library's implementation can only
implement optimizations that work in the most general case.

In this paper, we describe a source to source code transformation
system that automatically widens an existing library's public
interface, and then uses memoization to implement application specific
optimizations.  We argue that while such a transformation could be
applied manually, doing so would make it difficult to maintain the
library, and would convolute client code, decreasing the usablity of
the system.
}

\vspace*{-18pt}


\section{Introduction}
\label {intro}

\rcs{Explain what LLADD is here, and then describe the techniques used by the optimizer}

\section{Prior work}
\label {prior}

\rcs{We need to look into this!}

\section{System Design}

\section{Cil and Dynamic Checks}

\section{Instrumenting for Blast}

\section{Evaluation}

\subsection{Performance of transformed code}

\subsection{Number of dynamic checks removed by Blast}

\section{Conclusion}

\begin{thebibliography}{99}
\begin{small}
\bibitem[1]{multipleGenericLocking} Agrawal, et al. {\em Concurrency Control Performance Modeling: Alternatives and Implications}. TODS 12(4): (1987) 609-654

%\bibitem[2]{bdb} Berkeley~DB, {\tt http://www.sleepycat.com/}

%\bibitem[3]{capriccio} R. von Behren, J Condit, F. Zhou, G. Necula, and E. Brewer. {\em Capriccio: Scalable Threads for Internet Services} SOSP 19 (2003).

\bibitem[2]{oo7} Carey, Michael J., DeWitt, David J., Naughton, Jeffrey F. {\em The OO7 Benchmark.} SIGMOD (1993)

\bibitem[3]{relational} E. F. Codd, {\em A Relational Model of Data for Large Shared Data Banks.} CACM 13(6) p. 377-387 (1970)

\bibitem[4]{mapReduce} Jeffrey Dean and Sanjay Ghemawat. {\em Simplified Data Processing on Large Clusters. } OSDI (2004)

%\bibitem[5]{lru2s} Envangelos P. Markatos. {\em On Caching Search Engine Results}.  Institute of Computer Science, Foundation for Research \& Technology - Hellas (FORTH) Technical Report 241 (1999)

\bibitem[5]{soft-updates} Greg Ganger.  {\em Soft Updates: A Solution to the Metadata Update Problem in File Systems } ACM Transactions (2000)

\bibitem[6]{semantic} David K. Gifford, P. Jouvelot, Mark A. Sheldon, and Jr. James W. O'Toole. {\em Semantic file systems}. Proceedings of the Thirteenth ACM Symposium on Operating Systems Principles, (1991) p. 16-25.

\bibitem[7]{physiological} Gray, J. and Reuter, A. {\em Transaction Processing: Concepts and Techniques}. Morgan Kaufmann (1993) San Mateo, CA

\bibitem[8]{hierarcicalLocking} Jim Gray, Raymond A. Lorie, and Gianfranco R. Putzulo. {\em Granularity of locks and degrees of consistency in a shared database}. In 1st International Conference on VLDB, September 1975. Reprinted in Readings in Database Systems, 3rd ed.

\bibitem[9]{cht}  Gribble, Steven D., Brewer, Eric A., Hellerstein, Joseph M., Culler, David.  {\em Scalable, Distributed Data Structures for Internet Service Construction. } OSDI (2000)

\bibitem[9]{haerder} Haerder \& Reuter {\em "Principles of Transaction-Oriented Database Recovery." } Computing Surveys 15(4) (1983) % p 287-317

\bibitem[10]{hibernate} Hibernate, {\tt http://www.hibernate.org/}

\bibitem[11]{lamb} Lamb, et al., {\em The ObjectStore System.} CACM 34(10) (1991)

%\bibitem[12]{blink} Lehman \& Yao, {\em Efficient Locking for Concurrent Operations in B-trees.} TODS 6(4) (1981) p. 650-670

\bibitem[12]{lht} Litwin, W., {\em Linear Hashing: A New Tool for File and Table Addressing}. Proc. 6th VLDB, Montreal, Canada, (Oct. 1980) % p. 212-223

\bibitem[13]{aries} Mohan, et al., {\em ARIES: A Transaction Recovery Method Supporting Fine-Granularity Locking and Partial Rollbacks Using Write-Ahead Logging.} TODS 17(1) (1992) p. 94-162

\bibitem[14]{twopc} Mohan, Lindsay \& Obermarck, {\em Transaction Management in the R* Distributed Database Management System} TODS 11(4) (1986) p. 378-396

\bibitem[15]{ariesim} Mohan, Levine. {\em ARIES/IM: an efficient and high concurrency index management method using write-ahead logging} International Converence on Management of Data, SIGMOD (1992) p. 371-380

\bibitem[16]{mysql} {\em MySQL}, {\tt http://www.mysql.com/ }

\bibitem[17]{reiser} Reiser,~Hans~T. {\em ReiserFS 4} {\tt http://www.namesys.com/ }
%
\bibitem[18]{berkeleyDB} M. Seltzer, M. Olsen. {\em LIBTP: Portable, Modular Transactions for UNIX}. Proceedings of the 1992 Winter Usenix (1992)

\bibitem[19]{lrvm} Satyanarayanan, M., Mashburn, H. H., Kumar, P., Steere, D. C., AND Kistler, J. J. {\em Lightweight Recoverable Virtual Memory}. ACM Transactions on Computer Systems 12, 1 (Februrary 1994) p. 33-57. Corrigendum: May 1994, Vol. 12, No. 2, pp. 165-172.

\bibitem[20]{newTypes} Stonebraker. {\em Inclusion of New Types in Relational Data Base. } ICDE (1986) %p. 262-269

\bibitem[21]{postgres} Stonebraker and Kemnitz. {\em The POSTGRES Next-Generation Database Management System. } CACM (1991)

%\bibitem[SLOCCount]{sloccount} SLOCCount, {\tt http://www.dwheeler.com/sloccount/ }
%
%\bibitem[lcov]{lcov} The~LTP~gcov~extension, {\tt http://ltp.sourceforge.net/coverage/lcov.php }
%


%\bibitem[Beazley]{beazley} D.~M.~Beazley and P.~S.~Lomdahl, 
%{\em Message-Passing Multi-Cell Molecular Dynamics on the Connection
%Machine 5}, Parall.~Comp.~ 20 (1994) p. 173-195.
%
%\bibitem[RealName]{CitePetName} A.~N.~Author and A.~N.~Other, 
%{\em Title of Riveting Article}, JournalName VolNum (Year) p. Start-End
%
%\bibitem[ET]{embed} Embedded Tk, \\
%{\tt ftp://ftp.vnet.net/pub/users/drh/ET.html}
%
%\bibitem[Expect]{expect} Don Libes, {\em Exploring Expect}, O'Reilly \& Associates, Inc. (1995).
%
%\bibitem[Heidrich]{heidrich} Wolfgang Heidrich and Philipp Slusallek, {\em
%Automatic Generation of Tcl Bindings for C and C++ Libraries.},
%USENIX 3rd Annual Tcl/Tk Workshop (1995).
%
%\bibitem[Ousterhout]{ousterhout} John K. Ousterhout, {\em Tcl and the Tk Toolkit}, Addison-Wesley Publishers (1994).
%
%\bibitem[Perl5]{perl5} Perl5 Programmers reference,\\
%{\tt http://www.metronet.com/perlinfo/doc}, (1996).
%
%\bibitem[Wetherall]{otcl} D. Wetherall, C. J. Lindblad, ``Extending Tcl for
%Dynamic Object-Oriented Programming'', Proceedings of the USENIX 3rd Annual Tcl/Tk Workshop (1995).
\end{small}
\end{thebibliography}



\end{document}
